\documentclass[letterpaper,11pt]{article}

\usepackage{tocloft}
\usepackage{anyfontsize}
\usepackage{hyperref}
\usepackage{geometry}
\usepackage{tgheros}

\raggedright
\pagenumbering{gobble}

\geometry{
    a4paper,
    total={210mm,297mm},
    margin=0.04in,
    left=0.45in,
    right=0.45in,
    top=0.5in,
    bottom=0.4in,
}

\begin{document}
{\fontfamily{qhv}\selectfont
\begin{tabular*}{7.5in}{l@{\extracolsep{\fill}}r}
    \textbf{\Large Kunal Kapila} & \href{mailto:kunalkap@iitk.ac.in}{kunalkap@iitk.ac.in}\\
    Bachelor of Science, Mathematics and Scientific Computing & \href{http://home.iitk.ac.in/~kunalkap/}{home.iitk.ac.in/$\sim$kunalkap/}\\
    Double Major, Computer Science and Engineering & +91-9005685064\\
    \hline
\end{tabular*}
\vspace{5pt}

\Large{\textbf{Education}}
\small
\begin{itemize}
    \item
        \textbf{Indian Institute of Technology, Kanpur} \hfill \textit{2014 - 2019 (Expected)}\\
        Cumulative Grade Point Average 8.9/10\\
    \item
        \textbf{All India Senior School Certificate Examination (AISSCE)}\\
        Central Board of Secondary Education (CBSE), Govt. of India; Delhi Public School, R. K. Puram\\
        \begin{itemize}
            \item 98.2\% (School Topper) with National top 1\% in Computer Science \& Chemistry \hfill \textit{2014}\\
            \item Cumulative Grade Point Average (CGPA) 10/10 \hfill\textit{2012}\\
        \end{itemize}
\end{itemize}

\Large{\textbf{Academic and Co-Curricular Achievements}}
\small

\begin{itemize}
    \item Secured AIR 116 in \textbf{JEE Mains} 2014 amongst 1.5 million candidates
    \item Secured AIR 1306 in \textbf{JEE Advanced} 2014 amongst 150,000 candidates
    \item Secured Rank 29 in \textbf{ACM-ICPC Amritapuri} Onsite Round 2015 amongst 200 college teams that qualified for it
    \item Secured Rank 54 in \textbf{ACM-ICPC Amritapuri} Online Round 2015 amongst 1500 college teams
    \item \textbf{Microsoft Code.Fun.Do}
        \begin{itemize}
            \item \textbf{Winner 2014-15}: Graphing app to plot implicit 2D Mathematical Functions for Windows Phone Platform
            \item \textbf{National 5$^{th}$, Coding Milestone, Finalist Forum 2014-15}: Educational app to teach students programming on the go for Windows Phone Platform; Implemented Scratchpad, Lessons \& Daily Challenges
            \item \textbf{Winner 2015-16}: Front-end for open source program simulation code for Microsoft Windows Store
        \end{itemize}
    \item Qualified for \& attended \href{http://www.iarcs.org.in/inoi/2014/inoi2014/results_inoi2014.php}{\textbf{International Olympiad of Informatics Training Camp} 2014}
    \item \textbf{Kishore Vaigyanik Protsahan Yojna} Fellow 2013, Department of Science \& Technology, Government of India
    \item \textbf{Junior Science Talent Search Examination} Scholar 2012, Directorate of Education, Government of Delhi
    \item \textbf{National Talent Search Examination} Scholar 2010, conducted by NCERT, Government of India
\end{itemize}

\Large{\textbf{Work Experience}}
\small

\begin{itemize}
    \item Development Intern, \textbf{IITK NYC Office} \hfill \textit{MAY 2016 - current}\\
    \textit{\href{http://www.cse.iitk.ac.in/users/manindra/}{Supervisor: Prof. Manindra Agrawal, Department of CSE, IIT Kanpur}}
        \begin{itemize}
            \item Working on the backend of a scale-able web application using Scala with Akka
            \item Currently working on an efficient way of implementing history for editable items in a stream of objects
        \end{itemize}
    \item Web Developer (Intern), \href{http://elanic.in//}{\textbf{Elanic}, Bangalore} \hfill \textit{DEC 2015}
        \begin{itemize}
            \item Full stack developer: MongoDB, AngularJS and NodeJS along with Restify framework
            \item Built RESTful APIs for notification, engagement(likes, comments) and search (using Elastic Search) modules
        \end{itemize}
\end{itemize}

\Large{\textbf{Projects}}
\small

\begin{itemize}
    \item \textbf{PhotoProof}: Cryptographic Image Authentication \hfill \textit{JUN 2016 - current}\\
    \href{http://engineering.nyu.edu/people/nasir-memon}{\textit{Supervisor: Prof. Nasir Memon, (Department Head) Department of CSE, NYU School of Engineering}}
    \begin{itemize}
        \item Understood the theoretical model for image authentication based on Proof-Carrying Data (PCD)
        \item Currently working on a practical implementation of the idea described in the paper using libsnark's PCD
    \end{itemize}
    \item \textbf{Cimulator}: a Visual \& Interactive Tutoring system\hspace{0.2in} \href{http://home.iitk.ac.in/~kunalkap/Cimulator.html}{\textit{Click here for \textbf{Tutorial}}} \hfill \textit{MAY - JUL 2015} \\
    \href{http://www.cse.iitk.ac.in/users/karkare/}{\textit{Supervisor: Prof. Amey Karkare, Department of CSE, IIT Kanpur}}\hfill \textit{Tested by 400 students in Fall} 2015\\
        \begin{itemize}
            \item Interpreter for C language in python, modelled memory artificially, handled external header files \& overflows
            \item Prompts user with possible corrections for runtime errors instead of crashing abruptly (unlike gcc)
            \item Web interface to \textbf{simulate C codes} that were interpreted by Cimulator to provide visual cues to the user
        \end{itemize}
\end{itemize}

\Large{\textbf{Technical Skills}}
\small
\begin{itemize}
    \item \textbf{Languages}:\\
        C (\textit{Proficient}), C++ (\textit{Expert}), Python (\textit{Expert})
    \item \textbf{Web Development}:\\
        HTML, CSS, Javascript, Bootstrap, PHP, Node.js (with Express, Restify), AngularJS, MySQL, MongoDB
    \item \textbf{Tools}:\\
        \LaTeX, Git, Vim, OpenGL, SDL, GDB, MATLAB
        % \item \textbf{Operating Systems}: Arch Linux, Ubuntu, Windows
        %\item \textbf{Development}: Android app development, Windows app development
\end{itemize}

\pagebreak

\Large{\textbf{Course Projects}}
\small
\begin{itemize}
    \item \textbf{Animation Movie} \hfill \textit{OCT - NOV 2015}\\
    \href{https://sites.google.com/site/vinaynamboodiri/}{\textit{Supervisor: Prof. Vinay P. Namboodari, Department of CSE, IIT Kanpur}}
    \begin{itemize}
        \item Rendered a 3-minute movie clip using \textbf{OpenGL} (Graphics Library), \textbf{SDL} (Simple DirectMedia Layer)
        \item Implemented Keyframe animation, DeBoor's B-Spline Interpolation, Texture Mapping, Lighting, Physics-based collision detection, Dynamic Programming and integrated audio
    \end{itemize}
    \item \textbf{Randomness Certification using Quantum Non-Locality} \hfill \textit{FEB - APR 2016}
        \\
        \href{http://www.cse.iitk.ac.in/users/rmittal/}{\textit{Supervisor: Prof. Rajat Mittal, Department of CSE, IIT Kanpur}}
        \begin{itemize}
            \item Understand Quantum non-locality (Bell inequalities and non local games), and use this knowledge to study randomness certification, and a protocol to generate random strings of numbers
            \item \href{http://home.iitk.ac.in/~kunalkap/CS682_PreProjectReport.pdf}{Click \textbf{here} to view the preliminary project report}
        \end{itemize}
    \item \textbf{Students' Gymkhana Form Automation} \hfill \textit{MAR - APR 2016}\\
        \href{http://www.cse.iitk.ac.in/users/sganguly/}{\textit{Supervisor: Prof. Sumit Ganguly, Department of CSE, IIT Kanpur}} \hfill \textit{\href{https://github.com/kunalapila/SSF}{Code: Github Link}}
        \begin{itemize}
            \item Automate a commonly used Students' Gymkhana form: Senator Seed Fund
            % \item \href{http://home.iitk.ac.in/~kunalkap/CS315_SGFormsAutomation.pdf}{Click \textbf{here} to view the preliminary project report}
        \end{itemize}
        % ==================

\end{itemize}

\Large{\textbf{Relevant Courses Undertaken}}
\small

\begin{tabular}{p{4cm} p{7cm} p{7cm}}
\\
\textbf{ Mathematics} & Mathematics I (Functional Analysis) & Mathematics II (Linear Algebra) \vspace{4pt}\\
& Probability \& Statistics & Mathematical Logic\vspace{4pt}\\
& Abstract Algebra & Commutative Algebra (*)\vspace{4pt}\\
& Real Analysis & Introduction to Fourier Series (*)\vspace{4pt}\\
& Complex Analysis & Several Variable Calculus (*)\vspace{9pt}\\

\textbf{ Computer Science} & Fundamentals of Computing (A*) & Data Structures and Algorithms (A*)\vspace{4pt}\\
& Algorithms II (*) & Theory of Computation (*)\vspace{4pt}\\
& Modern Cryptology & Quantum Computing\vspace{4pt}\\
& Principles of Database Systems & Introduction to Computer Graphics\vspace{9pt}\\

\textbf{ Others} & Introduction to Economics (A*) & Introduction to Philosophical Logic\vspace{9pt}\\
\end{tabular}

\footnotesize{\hspace*{0.5in}(A*) Awarded for exceptional performance
\hspace{1.5in}  (*) Courses in Progress\\

\vspace{10pt}

\Large{\textbf{Positions of Responsibility}}
\small

\begin{itemize}
    \item \textbf{Coordinator, Programming Club}, Science and Technology Council, IIT Kanpur \hfill \textit{MAR 2016 - current}
        \begin{itemize}
            \item Conducted several lecture series, workshops and competitions on Algorithmic Programming, Open Source Development, and Web Development for students of the campus community
            \item Organised the summer camp and mentored over 60 students who completed projects in the field of Web \& App Development, Augmented Reality, Machine Learning, Computer Vision and Ethical Hacking
        \end{itemize}
    \item \textbf{Head Web, Antaragni}, Students' Gymkhana, IIT Kanpur \hfill \textit{MAR 2016 - current}
        \begin{itemize}
            \item Conceptualised and designed the website for Antaragni 2016
                (\href{http://antaragni.in/}{\textit{http://antaragni.in/}})
            \item Implemented NodeJS server with Express framework, along with MongoDB \& AngularJS
            \item Dynamically rendered the events, contacts, sponsors page and schedule from the database
        \end{itemize}
    \item \textbf{Student Member, Senate Undergraduate Committee (SUGC)}, IIT Kanpur \hfill \textit{(2014-15) \& (2015-16)}
    \begin{itemize}
        \item Amongst the \textbf{4} student members responsible for representing the opinions of more than 3500 undergraduate students in campus on academic matters in SUGC, a standing committee of the Institute Senate
        \item Worked on proposals like remedial programme for academically deficient students, Undergraduate Teaching Assistants \& Lateral Entry for Bachelors programme in Physics, Chemistry and Mathematics
    \end{itemize}
\end{itemize}
}
\end{document}
